\documentclass[a4paper, 12pt, diplomski]{etfcyr}


\usepackage{fontspec}
\usepackage{polyglossia}
\usepackage[document]{ragged2e}
\usepackage{indentfirst}
\usepackage{enumitem}
\usepackage{scrextend}
\usepackage{etoolbox}

\addtokomafont{labelinglabel}{\textbf}

\setdefaultlanguage[script=Cyrillic]{serbian}
\newfontfamily{\cyrillicfont}{Times New Roman}

\setmainfont{Times New Roman}
\setsansfont{Arial}
\setmonofont{Courier New}

\renewcommand*\contentsname{Садржај}

\newcommand{\indentfirstparagraphon}{
    \renewenvironment{justify}{%
    	\trivlist
    	\justifying
    	\itemindent\JustifyingParindent
    	\item\relax
    }{
        \endtrivlist
    }
}

\newcommand{\indentfirstparagraphoff}{
    \renewenvironment{justify}{%
        \trivlist
        \justifying
        \item\relax
    }{
        \endtrivlist
    }
}

\makeatletter
\gdef\tshortstack{\@ifnextchar[\@tshortstack{\@tshortstack[c]}}
\let\@tshortstack\@shortstack
\patchcmd\@tshortstack\vbox\vtop{}{}
\makeatother

\indentfirstparagraphon

\addto\captionsserbian{%
  \renewcommand{\bibname}%
    {Литература}%
}

\setlength{\parindent}{4em}
\setlength{\JustifyingParindent}{4em}

\title{Паметна соба - контрола уласка} 
\author{Топлица Танасковић}
\indeks{164/1996}
\date{септембар 2016.}
\mentor{Саша Стојановић}
\predmet{Системско програмирање}

\begin{document}
	\sloppy
	\maketitle
	\begin{abstract}
		\begin{justify}
			У овом раду је изложен проблем прављења распореда коришћења лабораторија са контролом уласка. Објашњени су најчешћи сценарији који задају доста проблема и предложено је софтверско решење.
		\end{justify}
	\end{abstract}

	\begin{keywords}
		Дипломски радови, контрола приступа, програмирање, C++
	\end{keywords}
	\tableofcontents
	\listoffigures
	\listoftables

	\chapter{Увод}
		
		\section{Опис проблема}
		
		\begin{justify}
			Прављење распореда рада у факултетским лабораторијама је велики проблем са којим особље мора стално да се носи. Број студената и актвности које морају да се обаве је огроман, а факултет има ограничен број и лабораторија и радних места у њима. Ово представља проблем за људе који су одговорни за прављење распореда пошто морају да воде рачуна о много активности, студената и временских слотова док у исто време покушавају да избегну преклапања или конфликте.
		
			Проблем можемо сагледати са два аспекта, први је прављење распореда, а други је безбедност. Како је већ напоменуто, факултет има ограничен број лабораторија одређеног капацитета док је број студената са својим задацима и обавезним активностима огроман.
		
			Прављење распореда рада са дозволом уласка је врло тешко јер захтева праћење великог броја, како студената, тако и активности како би се обезбедило да нема преклапања. Радити ово ручно није оптимално и подложно је грешкама што може да доведе до разних преклапања, промена распореда у задњи час и многих других проблема. Без обзира што многе активности нису обавезне доста активности је у исто време и обавезно и врло битно за студенте, као што су рецимо колоквијуми или испити. Овакве активности не би смеле да се одлажу нити прекидају улажењем других особа у просторију.
		
			Што се тиче безбедности, неке лабораторије осим што имају деликатну и скупу, имају и врло опасну опрему која може услед погрешног руковања да начини огромну штету или чак нанесе опасне повреде. Због овога би требало ограничити приступ тј. улаз у неке од лабораторија како би се избегле у крајњим случајевима и несагледиве последице.
		\end{justify}

	\chapter{Дискусија}
		
		\section{Начини коришћења лабораторија}
		
		\begin{justify}
			Број начина на које једна лабораторија може да се користи углавном зависи од њене намене. Међутим, без обзира на намену лабораторије могу се уочити четири најчешћа начина на које се оне користе. Да би се целокупан проблем прецизније дефинисао, у овом одељку ће се дискутовати понаособ сваки од начина коришћења.
		\end{justify}
		
		\subsection{Дефиниције}
		
		\begin{justify}
			Пре него што будемо могли да опишемо начине на које се лабораторије користе, морамо увести неке дефиниције и успоставити релације између њих.
		\end{justify}


   		\begin{labeling}{\smash{\tshortstack[l]{Корисник\\лабораторије}}}
            \indentfirstparagraphoff            
            
            \item [Активност] 
                \begin{justify}
                    Активност је било који процес који се одвија у лабораторији, без обзира на његово трајање, евентуалну периодичност или конкретан тип.
                \end{justify}

            \item[\smash{\tshortstack[l]{Тип\\активности}}]
                \begin{justify}
                   Тип активности ближе дефинише активност, и одређује њен приоритет у односу на друге. Постоје следећи типови активности.
                   \begin{enumerate}[noitemsep]
                       \item Испит
                       \item Колоквијум
                       \item Предавања или вежбе
                       \item Лабораторијске вежбе
                       \item Посебни догађаји
                       \item Индивидуални рад
                    \end{enumerate}
                \end{justify}

            \item[\smash{\tshortstack[l]{Приоритет\\активности}}]
                \begin{justify}
                Приоритет активности одређује да ли је дозвољен улаз у просторију уколико се у њој преклапају активности, а једна од њих је битнија. Не желимо да допустимо да људи улазе и излазе из просторије док је у току испит. Приоритет активност је дефинисан типом активности како је наведено у претходној дефиницији.

                \indent Индивидуални рад је посебан случај. Приликом заказивања такве активности може се посебно дозволити улазак без обзира на то да ли се у просторији већ одвија нека активност вишег приоритета. Ово се углавном користи за студенте који раде дипломски или неки други важан пројекат.
                
                \end{justify}


            \item[\smash{\tshortstack[l]{Корисник\\лабораторије}}]
                \begin{justify}
                    Корисник лабораторије је било која особа којој се може дозволити улаз у лабораторију. Постоје две врсте корисника:
                    \begin{enumerate}[noitemsep]
                        \item Професор
                        \item Студент
                    \end{enumerate}
                    Можда би се могао увести и студент демонстратор као засебан тип корисника, али како он нема посебна права, довољно је да буде члан листе како би могао да приступи лабораторији.
                \end{justify}
            
            \item [Листа]
                \begin{justify}
                    Листе су групе особа-корисника лабораторија које служе зарад лакше манипулације током заказивања активности и доделе права уласка у лабораторију.
                    \begin{enumerate}[noitemsep]
                        \item Системске листе
                        \item Трајне листе
                        \item Привремене листе
                    \end{enumerate}
                    Привремене листе имају свој рок трајања и након тога се аутоматски бришу из система. Системске листе су посебан случај трајних листа, оне се не могу мењати и о њима систем сам води рачуна. Пример системске листе је листа студената друге године.
                \end{justify}
            
            \item [Просторија]
                \begin{justify}
                    Просторија је било која лабораторија, кабинет или учионица у којој се одвијају неке активности у за коју желимо да имамо контролу уласка.
                \end{justify}
            
            \indentfirstparagraphon
    	\end{labeling}
        
\end{document}