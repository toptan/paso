\documentclass[a4paper, 12pt, diplomski]{etfcyr}


\usepackage{fontspec}
\usepackage{polyglossia}
\usepackage[document]{ragged2e}
\usepackage{indentfirst}

\setdefaultlanguage[script=Cyrillic]{serbian}
\newfontfamily{\cyrillicfont}{Times New Roman}

\setmainfont{Times New Roman}
\setsansfont{Arial}
\setmonofont{Courier New}

\renewcommand*\contentsname{Садржај}
\renewenvironment{justify}{%
	\trivlist
	\justifying
	\itemindent\JustifyingParindent
	\item\relax
}{%
\endtrivlist
}
\addto\captionsserbian{%
  \renewcommand{\bibname}%
    {Литература}%
}
\setlength{\parindent}{4em}
\setlength{\JustifyingParindent}{4em}

\title{Паметна соба - контрола уласка} 
\author{Топлица Танасковић}
\indeks{164/1996}
\date{август 2016.}
\mentor{Саша Стојановић}
\predmet{Системско програмирање}

\begin{document}
	\sloppy
	\maketitle
	\begin{abstract}
		\begin{justify}
			У овом раду је изложен проблем прављења распореда коришћења лабораторија са контролом уласка. Објашњени су најчешћи сценарији који задају доста проблема и предложено је софтверско решење.
		\end{justify}
	\end{abstract}

	\begin{keywords}
		Дипломски радови, контрола приступа, програмирање, C++
	\end{keywords}
	\tableofcontents
	\listoffigures
	\listoftables

	\chapter{Увод}
		
		\section{Опис проблема}
		
		\begin{justify}
			Прављење распореда рада у факултетским лабораторијама је велики проблем са којим особље мора стално да се носи. Број студената и актвности које морају да се обаве је огроман, а факултет има ограничен број и лабораторија и радних места у њима. Ово представља проблем за људе који су одговорни за прављење распореда пошто морају да воде рачуна о много активности, студената и временских слотова док у исто време покушавају да избегну преклапања или конфликте.
		
			Проблем можемо сагледати са два аспекта, први је прављење распореда, а други је безбедност. Како је већ напоменуто, факултет има ограничен број лабораторија одређеног капацитета док је број студената са својим задацима и обавезним активностима огроман.
		
			Прављење распореда рада са дозволом уласка је врло тешко јер захтева праћење великог броја, како студената, тако и активности како би се обезбедило да нема преклапања. Радити ово ручно није оптимално и подложно је грешкама што може да доведе до разних преклапања, промена распореда у задњи час и многих других проблема. Без обзира што многе активности нису обавезне доста активности је у исто време и обавезно и врло битно за студенте, као што су рецимо колоквијуми или испити. Овакве активности не би смеле да се одлажу нити прекидају улажењем других особа у просторију.
		
			Што се тиче безбедности, неке лабораторије осим што имају деликатну и скупу, имају и врло опасну опрему која може услед погрешног руковања да начини огромну штету или чак нанесе опасне повреде. Због овога би требало ограничити приступ тј. улаз у неке од лабораторија како би се избегле у крајњим случајевима и несагледиве последице.
		\end{justify}
	
\end{document}