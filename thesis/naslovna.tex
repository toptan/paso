\documentclass[12pt]{article}

\usepackage{fontspec}
\usepackage{polyglossia}
\usepackage[document]{ragged2e}

\setmainfont{Times New Roman}
\setsansfont{Arial}
\setmonofont{Courier New}

\renewcommand*\contentsname{Садржај}

\begin{document}

	\begin{center}

		\Large Универзитет у Београду

		\vspace{16pt}

		Електротехнички факултет
		
		\vspace{144pt}
		
		ДИПЛОМСКИ РАД
		
		\vspace{64pt}
		
		\Huge Паметна соба - контрола уласка
		
		\vspace{64pt}
		
		\Large Топлица Танасковић 1996/0164
		
		\vfill
		
		Београд, 2016.
	\end{center}
	\pagebreak
	
	\tableofcontents
	
	\pagebreak
	
	\normalsize
	\setlength{\parindent}{4em}
	\setlength{\JustifyingParindent}{4em}
	
	\section{Увод}
	
	\begin{justify}
		
		Прављење распореда рада у факултетским лабораторијама је велики проблем са којим особље мора стално да се носи. Број студената и актвности које морају да се обаве је огроман, а факултет има ограничен број и лабораторија и радних места у њима. Ово представља проблем за људе који су одговорни за прављење распореда пошто морају да воде рачуна о много активности, студената и временских слотова док у исто време покушавају да избегну преклапања или конфликте.
		
		Овај рад дефинише проблем детаљније, излаже неке уобичајене сценарије и описује једно од решења.
	\end{justify}
		
	\section{Проблем}
	
	\begin{justify}
		
		Проблем можемо сагледати са два аспекта, прављење распореда и безбедност. Како је већ напоменуто, факултет има ограничен број лабораторија одређеног капацитета док је број студената са својим задацима и обавезним активностима огроман.
		
		Прављење распореда рада и дозвољавање уласка је врло тешко јер захтева праћење великог броја, како студената, тако и активности како би се обезбедило да нема преклапања. Радити ово ручно није оптимално и подложно је грешкама што може да доведе до разних преклапања, промена распореда у задњи час и многих других проблема. Без обзира што многе активности нису обавезне доста активности је у исто време и обавезно и врло битно за студенте, као што су рецимо колоквијуми или испити. Овакве активности не би смеле да се одлажу нити прекидају улажењем других особа у просторију.
		
		Други аспект је безбедност. Неке лабораторије осим што имају деликатну имају и врло опасну опрему која може услед погрешног руковања да начини огромну штету или чак нанесе опасне повреде.
	\end{justify}
	
	
\end{document}